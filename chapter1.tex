\chapter{Background}

\section{Heart Disease}
\lipsum[1]

\section{Standard Cardiac Magnetic Resonance Imaging (MRI) and Traditional Measures of Cardiac Function}
Magnetic resonance imaging (MRI) is a non-invasive non-ionizing medical imaging technique that is used to assess the function and health of internal structures of the human body. MRI involves the use of strong magnetic fields and radio waves along with the natural hydrogen nuclei, which are abundant in human tissue, to generate images. MRI can be used to non-invasively assess the cardiac function and health. Generating images of the heart is similar to MRI except for a few changes due to the continuous movement of the heart itself, such as electrocardiogram or ECG gating.

Non-invasive imaging has become standard protocol in the diagnosis and management of heart diseases, such as CHD [cite]. Traditional measures of cardiac function, such as ventricular volumes and mass or ejection fraction, can be derived from standard cardiac MRI. Whole heart function is typically assessed with these traditional metrics, but unfortunately, they may not contain enough information to explain the complex nature of some heart diseases. Moreover, there is a growing body of evidence that suggests that, when combined with clinical risk factors (e.g. hypertension), advanced measures of cardiac mechanics (e.g. cardiac strains and torsion) are better predictors of mortality compared to traditional measures \cite{Stanton2009}.

- put stanton figure in introduction

\section{Advanced Measures of Function Cardiac Mechanics}
 - explain what they are., explain why they are important, explain that DENSE can be used to acquire the data to derive them

\section{Displacement Encoded Cardiac MRI}
Spiral cine Displacement ENcoding with Stimulated Echoes (DENSE) is an advanced cardiac magnetic resonance imaging technique that directly encodes the displacement of the myocardial tissue into the phase of the MR signal [cite]. Because of its quantitative nature, as opposed to qualitative, such as in standard cardiac MRI, it allows for simple quantification of cardiac mechanics. In addition, DENSE has good spatial resolution and good reproducibility [cite].

\section{Respiratory Motion and Blurring}
-images are typically acquired using breath-holds
-without breath-holds, blurring occurs
-dense requires a lot of data that cannot be acquired during a single breath-hold
-in order to derive complext information or to acquire large amounts of data, respiratory navigator gating is required

\section{Respiratory Navigator Gating}
- respiratory navigator gating works by...it's good because you can get a lot of data within a scan, but the issue is it takes too long. Moreover, it is unknown how it affects the measures of cardiac mechanics.

\section{Dissertation Outline}
\lipsum[6]

\index In Chapters 2 and 3, we address the effects of inconsistent end-expiratory diaphragm position between breath-holds and respiratory navigator gating on DENSE-derived cardiac mechanics, such as left ventricular strain and torsion. In Chapter 2, we learn that cardiac strain is insensitive to normal changes in end-expiratory position between breath-hold DENSE acquisitions. In Chapter 3, we discover that use of a respiratory navigator has the ability to significantly reduce the variability of cardiac torsion and thus the sample size needed to detect small changes in torsion. The conclusions of the studies performed for Chapters 2 and 3 demonstrate the importance of employing a respiratory navigator or some form of consistent respiratory compensation for future studies.

\index In Chapter 4, we address the 