\begin{appendices}
	\appendixtitletocoff
	\appendix
	
\chapter{Differences in Segmental Strain Between Different Acceptance Window Positions}
	
	\begin{sidewaystable}
		\centering
		\caption[Segmental circumferential strain (\%, mean $\pm$ standard deviation) from the three acceptance window positions (minimum, middle, and maximum) for all subjects combined]{\textbf{Segmental circumferential strain (\%, mean $\pm$ standard deviation) from the three acceptance window positions (minimum, middle, and maximum) for all subjects combined.}}
		\label{table:SegmentialEccStrainDiff}
		\begin{tabular}{ccccccccccccc}
			\toprule
			% Top Headings
			\multirow{2}{*}{} & \multicolumn{2}{c}{Anterior} & \multicolumn{2}{c}{Anteroseptal} & \multicolumn{2}{c}{Inferoseptal} &
			\multicolumn{2}{c}{Inferior} & \multicolumn{2}{c}{Inferolateral} & \multicolumn{2}{c}{Anterolateral}\\
			% Second from top headings
			 & \textbf{Strain} & \textbf{P} & \textbf{Strain} & \textbf{P} & \textbf{Strain} & \textbf{P} &
			   \textbf{Strain} & \textbf{P} & \textbf{Strain} & \textbf{P} & \textbf{Strain} & \textbf{P} \\
			\midrule
			
			% Basal Data
			\multicolumn{13}{c}{\textbf{Basal}} \\
			\midrule
			\textbf{Max} & -16$\pm$5 & \multirow{3}{*}{0.99} & -14$\pm$5 & \multirow{3}{*}{1.0} & -14$\pm$5 & \multirow{3}{*}{0.95}
			             & -15$\pm$5 & \multirow{3}{*}{0.91} & -19$\pm$6 & \multirow{3}{*}{0.88} & -19$\pm$5 & \multirow{3}{*}{0.76} \\
			\textbf{Mid} & -16$\pm$3 &                       & -13$\pm$5 &                       & -15$\pm$5 & 
						 & -15$\pm$5 &                       & -18$\pm$5 &                       & -19$\pm$5 &                       \\
			\textbf{Min} & -15$\pm$4 &                       & -13$\pm$5 &                       & -15$\pm$4 & 
						 & -15$\pm$5 &                       & -19$\pm$6 &                       & -18$\pm$4 &  \\ 
			\midrule
			
			% Mid-Ventricular Data
			\multicolumn{13}{c}{\textbf{Mid-Ventricular}} \\
			\midrule
			\textbf{Max} & -17$\pm$5 & \multirow{3}{*}{0.93} & -14$\pm$5 & \multirow{3}{*}{0.83} & -13$\pm$5 & \multirow{3}{*}{0.87}
						 & -17$\pm$5 & \multirow{3}{*}{0.93} & -22$\pm$6 & \multirow{3}{*}{1.0}  & -20$\pm$6 & \multirow{3}{*}{0.81} \\
			\textbf{Mid} & -17$\pm$5 &                       & -14$\pm$4 &                       & -14$\pm$4 & 
						 & -18$\pm$5 &                       & -21$\pm$4 &                       & -21$\pm$6 &                       \\
			\textbf{Min} & -17$\pm$6 &                       & -14$\pm$4 &                       & -13$\pm$4 & 
						 & -17$\pm$3 &                       & -20$\pm$5 &                       & -21$\pm$6 & \\ 
			\midrule
			
			% Apical Data
			\multicolumn{13}{c}{\textbf{Apical}} \\
			\midrule
			\textbf{Max} & -18$\pm$5 & \multirow{3}{*}{0.66} & -15$\pm$6 & \multirow{3}{*}{0.99} & -17$\pm$6 & \multirow{3}{*}{0.79}
						 & -20$\pm$6 & \multirow{3}{*}{0.98} & -23$\pm$7 & \multirow{3}{*}{0.84} & -22$\pm$6 & \multirow{3}{*}{0.99} \\
			\textbf{Mid} & -18$\pm$5 &                       & -15$\pm$5 &                       & -17$\pm$6 & 
						 & -20$\pm$6 &                       & -23$\pm$6 &                       & -22$\pm$7 &                       \\
			\textbf{Min} & -18$\pm$6 &                       & -15$\pm$5 &                       & -16$\pm$6 & 
						 & -21$\pm$6 &                       & -24$\pm$6 &                       & -21$\pm$7 & \\ 
			\bottomrule
			\multicolumn{13}{l}{P-values indicate results from test comparing acceptance window positions.}
		\end{tabular}
	\end{sidewaystable}

	\begin{sidewaystable}
		\centering
		\caption[Segmental radial strain (\%, mean $\pm$ standard deviation) from the three acceptance window positions (minimum, middle, and maximum) for all subjects combined]{\textbf{Segmental radial strain (\%, mean $\pm$ standard deviation) from the three acceptance window positions (minimum, middle, and maximum) for all subjects combined.}}
		\label{table:SegmentialErrStrainDiff}
		\begin{tabular}{ccccccccccccc}
			\toprule
			% Top Headings
			\multirow{2}{*}{} & \multicolumn{2}{c}{Anterior} & \multicolumn{2}{c}{Anteroseptal} & \multicolumn{2}{c}{Inferoseptal} &
			\multicolumn{2}{c}{Inferior} & \multicolumn{2}{c}{Inferolateral} & \multicolumn{2}{c}{Anterolateral}\\
			% Second from top headings
			& \textbf{Strain} & \textbf{P} & \textbf{Strain} & \textbf{P} & \textbf{Strain} & \textbf{P} &
			\textbf{Strain} & \textbf{P} & \textbf{Strain} & \textbf{P} & \textbf{Strain} & \textbf{P} \\
			\midrule
			
			% Basal Data
			\multicolumn{13}{c}{\textbf{Basal}} \\
			\midrule
			\textbf{Max} & 37$\pm$20 & \multirow{3}{*}{0.78} & 36$\pm$16 & \multirow{3}{*}{0.61} & 40$\pm$20 & \multirow{3}{*}{0.53}
					     & 43$\pm$23 & \multirow{3}{*}{0.77} & 47$\pm$29 & \multirow{3}{*}{0.94} & 41$\pm$21 & \multirow{3}{*}{0.64} \\
			\textbf{Mid} & 40$\pm$21 &                       & 41$\pm$18 &                       & 37$\pm$14 & 
					     & 36$\pm$21 &                       & 47$\pm$31 &                       & 44$\pm$24 &                       \\
			\textbf{Min} & 38$\pm$18 &                       & 38$\pm$17 &                       & 37$\pm$18 & 
						 & 40$\pm$22 &                       & 51$\pm$29 &                       & 46$\pm$28 &  \\ 
			\midrule
			
			% Mid-Ventricular Data
			\multicolumn{13}{c}{\textbf{Mid-Ventricular}} \\
			\midrule
			\textbf{Max} & 28$\pm$17 & \multirow{3}{*}{0.94} & 33$\pm$24 & \multirow{3}{*}{0.44} & 34$\pm$15 & \multirow{3}{*}{0.88}
						 & 32$\pm$15 & \multirow{3}{*}{0.96} & 36$\pm$32 & \multirow{3}{*}{0.83} & 32$\pm$17 & \multirow{3}{*}{0.49} \\
			\textbf{Mid} & 31$\pm$19 &                       & 35$\pm$19 &                       & 31$\pm$13 & 
					     & 33$\pm$24 &                       & 38$\pm$29 &                       & 33$\pm$18 &                       \\
			\textbf{Min} & 30$\pm$19 &                       & 36$\pm$14 &                       & 36$\pm$16 & 
						 & 31$\pm$21 &                       & 35$\pm$24 &                       & 28$\pm$17 &  \\ 
			\midrule
			
			% Apical Data
			\multicolumn{13}{c}{\textbf{Apical}} \\
			\midrule
			\textbf{Max} & 28$\pm$24 & \multirow{3}{*}{0.72} & 36$\pm$37 & \multirow{3}{*}{0.87} & 41$\pm$21 & \multirow{3}{*}{0.83}
						 & 41$\pm$37 & \multirow{3}{*}{0.20} & 31$\pm$23 & \multirow{3}{*}{0.84} & 27$\pm$16 & \multirow{3}{*}{0.69} \\
			\textbf{Mid} & 23$\pm$12 &                       & 31$\pm$13 &                       & 39$\pm$20 & 
						 & 40$\pm$20 &                       & 27$\pm$17 &                       & 26$\pm$15 &                       \\
			\textbf{Min} & 25$\pm$15 &                       & 35$\pm$24 &                       & 40$\pm$27 & 
						 & 34$\pm$24 &                       & 29$\pm$20 &                       & 33$\pm$16 &  \\ 
			\bottomrule
			\multicolumn{13}{l}{P-values indicate results from test comparing acceptance window positions.}
		\end{tabular}
	\end{sidewaystable}
	
	\chapter{Segmental Strains for Navigator Gating}
	\section{Peak Segmental Strains for Navigator Gating and Breath-holds}
	The myocardium was divided into 6 segments for mid-ventricular short-axis images and 7 segments for four-chamber long-axis images based on the American Heart Association 17-segment model. The peak strain was computed for each segment and reported using mean and standard deviation of all segments in Table S1. For four-chamber images, pixels from the most basal and apical segments (3 segments in total) were excluded from analysis in order to remove the increased noise typically observed in those regions, which also matched the analysis performed for global longitudinal strain.
	
	\begin{table}
		\centering
		\caption[Segmental strain results for navigator gating and breath-holds in adults]{\textbf{Segmental strain results for navigator gating and breath-holds in adults.}}
		\label{table:navigator_segmental_strains}
		\begin{tabular}{c  c}
			\toprule
			\multicolumn{1}{c}{}       			 & \multicolumn{1}{c}{\textbf{Mean $\pm$ Std. Dev.}} \\ 
			\multicolumn{1}{l}{\textbf{Circumferential Strain (\%)}} & \multicolumn{1}{c}{}          \\
			\multicolumn{1}{r}{Breath-hold}      & \multicolumn{1}{c}{-18 $\pm$ 5}                   \\
			\multicolumn{1}{r}{Retrospective}    & \multicolumn{1}{c}{-18 $\pm$ 5}                   \\
			\multicolumn{1}{r}{Prospective}      & \multicolumn{1}{c}{-18 $\pm$ 5}                   \\
			\multicolumn{1}{r}{Dual}    		 & \multicolumn{1}{c}{-18 $\pm$ 4}                   \\
			\multicolumn{1}{l}{\textbf{Radial Strain (\%)}}          & \multicolumn{1}{c}{}        	 \\
			\multicolumn{1}{r}{Breath-hold}      & \multicolumn{1}{c}{35 $\pm$ 16}                   \\
			\multicolumn{1}{r}{Retrospective}    & \multicolumn{1}{c}{34 $\pm$ 16}                   \\
			\multicolumn{1}{r}{Prospective}      & \multicolumn{1}{c}{42 $\pm$ 17}                   \\
			\multicolumn{1}{r}{Dual}		     & \multicolumn{1}{c}{34 $\pm$ 16}                   \\
			\multicolumn{1}{l}{\textbf{Longitudinal Strain (\%)}}    & \multicolumn{1}{c}{}          \\
			\multicolumn{1}{r}{Breath-hold}      & \multicolumn{1}{c}{-13 $\pm$ 4}                   \\
			\multicolumn{1}{r}{Retrospective}    & \multicolumn{1}{c}{-14 $\pm$ 3}                   \\
			\multicolumn{1}{r}{Prospective}      & \multicolumn{1}{c}{-13 $\pm$ 3}                   \\
			\multicolumn{1}{r}{Dual}		     & \multicolumn{1}{c}{-14 $\pm$ 4}            		 \\
			\bottomrule                                                 
		\end{tabular}
	\end{table}
	
	\section{Segmental Strain Agreement Between Navigator Gating and Breath-holds}
	Peak segmental strains were compared between each navigator gating configuration (retrospective, prospective, and dual) and breath-holds using Bland-Altman analyses and coefficient of variation (CoV). Comparisons were performed using a paired Student’s t-test.
	
	\begin{table}
		\centering
		\caption[Segmental strain agreement between navigator gating and breath-holds from spiral cine DENSE]{\textbf{Segmental strain agreement between navigator gating and breath-holds from spiral cine DENSE.}}
		\label{table:nav_CI_seg_strains}
		\begin{tabular}{c c c c c}
			\toprule
			\multicolumn{1}{c}{}&\multicolumn{1}{c}{\textbf{Bias}}&\multicolumn{1}{c}{\textbf{95\% LoA}}& \multicolumn{1}{c}{\textbf{CoV (\%)}} & \multicolumn{1}{c}{\textbf{p-value}}\\ \midrule
			\multicolumn{1}{l}{\textbf{Circumferential Strain (\%)}}                       						   & \multicolumn{4}{c}{}     \\
			\multicolumn{1}{r}{Retrospective--Breath-hold}  & \multicolumn{1}{c}{0}  & \multicolumn{1}{c}{$\pm$ 5} & \multicolumn{1}{c}{8}   & \multicolumn{1}{c}{0.94} \\
			\multicolumn{1}{r}{Prospective--Breath-hold}    & \multicolumn{1}{c}{0}  & \multicolumn{1}{c}{$\pm$ 8} & \multicolumn{1}{c}{13}  & \multicolumn{1}{c}{0.78} \\
			\multicolumn{1}{r}{Dual--Breath-hold}           & \multicolumn{1}{c}{-1} & \multicolumn{1}{c}{$\pm$ 5} & \multicolumn{1}{c}{8}   & \multicolumn{1}{c}{0.11} \\
			\multicolumn{1}{l}{\textbf{Radial Strain (\%)}}                       						   & \multicolumn{4}{c}{}     \\
			\multicolumn{1}{r}{Retrospective--Breath-hold}  & \multicolumn{1}{c}{0}  & \multicolumn{1}{c}{$\pm$ 25} & \multicolumn{1}{c}{19}   & \multicolumn{1}{c}{0.79} \\
			\multicolumn{1}{r}{Prospective--Breath-hold}    & \multicolumn{1}{c}{8}  & \multicolumn{1}{c}{$\pm$ 36} & \multicolumn{1}{c}{28}  & \multicolumn{1}{c}{0.002*} \\
			\multicolumn{1}{r}{Dual--Breath-hold}           & \multicolumn{1}{c}{-1} & \multicolumn{1}{c}{$\pm$ 29} & \multicolumn{1}{c}{23}   & \multicolumn{1}{c}{0.61} \\
			\multicolumn{1}{l}{\textbf{Longitudinal Strain (\%)}}                       						   & \multicolumn{4}{c}{}     \\
			\multicolumn{1}{r}{Retrospective--Breath-hold}  & \multicolumn{1}{c}{-1} & \multicolumn{1}{c}{$\pm$ 7} & \multicolumn{1}{c}{13}   & \multicolumn{1}{c}{0.17} \\
			\multicolumn{1}{r}{Prospective--Breath-hold}    & \multicolumn{1}{c}{0}  & \multicolumn{1}{c}{$\pm$ 9} & \multicolumn{1}{c}{17}   & \multicolumn{1}{c}{0.75} \\
			\multicolumn{1}{r}{Dual--Breath-hold}           & \multicolumn{1}{c}{-1} & \multicolumn{1}{c}{$\pm$ 8} & \multicolumn{1}{c}{13}   & \multicolumn{1}{c}{0.19} \\
			\bottomrule
			\multicolumn{5}{l}{\footnotesize* indicates p $<$ 0.05} \\                                 
		\end{tabular}
	\end{table}
	\chapter{Off-Scanner Training Protocol for Feedback Game}
	\section{Training Protocol}
	The goal-based training protocol was as follows: First, the children were instructed to perform 3 sequential end-expiratory breath-holds to determine the optimal location for the acceptance window. Then the children were instructed to complete 9 levels of the FG, which progressed in difficulty. Difficulty was increased by either 1) decreasing the acceptance window size or 2) increasing the time delay between chest excursion recording and fish location update. Because the navigator gating sequence only measures the diaphragm position during each heartbeat, children with slower heartbeats may experience "delays" between diaphragm movement and fish location update. In order to complete each level, the children had to acquire 100 points. If all bubbles were acquired in a row with no breaks, each level could be completed in $\sim$33 seconds.
	
	\section{Survey Responses}
	In order to formally measure the enjoyment and response of the children playing the Feedback Game, we asked the children to fill out a post-scan survey that consisted of 7 questions. Those questions and responses are listed below. In general, most participants 1) found Bubble Gulp to be easy to play; 2) enjoyed playing Bubble Gulp; 3) thought they were generally getting better as they played; 4) thought training was/would have been somewhat helpful; 5) had no comments on how to improve 'Bubble Gulp'; 6) enjoy playing videogames; and 7) play videogames daily.
	
	\begin{figure}
		\centering
		\includegraphics{figures/gamepaper/Question1}
		\caption[Survey Question 1]{Question 1.	How easy was playing Bubble Gulp? \\ 1: Really easy \\ 2: Easy \\3: Neither easy nor difficult \\4: Difficult \\5: Really Difficult}
		\label{fig:question1}
	\end{figure}

	\begin{figure}
		\centering
		\includegraphics{figures/gamepaper/Question2}
		\caption[Survey Question 2]{Question 2.	How much did you enjoy playing Bubble Gulp? \\ 1: Really enjoyed it \\ 2: Enjoyed it \\3: Neither \\4: Did not enjoy it \\5: Really did not enjoy it}
		\label{fig:question2}
	\end{figure}

	\begin{figure}
		\centering
		\includegraphics{figures/gamepaper/Question3}
		\caption[Survey Question 3]{Question 3.	Did you think you were getting better, stayed the same, or were getting worse as you were playing Bubble Gulp at the end of the study compared to when you first tried it? \\ 1: Better \\ 2: Stayed the same \\3: Worse}
		\label{fig:question3}
	\end{figure}

	\begin{figure}
		\centering
		\includegraphics{figures/gamepaper/Question4}
		\caption[Survey Question 4]{Question 4.	We have a pretend MRI scanner where you can learn to play Bubble Gulp before getting into the actual MRI scanner. Do you think using this pretend MRI scanner first would have been/was: \\ 1: Very helpful \\ 2: Somewhat helpful \\3: Not helpful \\ 4: A total wast of time}
		\label{fig:question4}
	\end{figure}

	\newpage
	5.	Do you have any comments on how to improve Bubble Gulp?
	\begin{itemize}
		\item Mostly "None"
		\item "Make the fish pink"
		\item "Make the lines further a part on the screen"
		\item "Liked the simple concept and how could control with breathing"
		\item "Reverse direction of fish movement with breathing"
		\item "Make not as glitchy, (make smoother)"
		\item "Make lines move to more comfortable spot to breathe in"
		\begin{itemize}
			\item This subject moved before their last scan
		\end{itemize}
	\end{itemize}
	
	\begin{figure}
		\centering
		\includegraphics{figures/gamepaper/Question6}
		\caption[Survey Question 6]{Question 6. How much do you enjoy playing videogames? \\ 1: Really enjoy\\ 2: Enjoy\\ 3: Neither\\ 4: Do not enjoy \\ 5: Really do not enjoy}
		\label{fig:question6}
	\end{figure}

	\begin{figure}
		\centering
		\includegraphics{figures/gamepaper/Question7}
		\caption[Survey Question 7]{Question 7. How often do you play videogames? \\ 1: Daily\\ 2: 2-3 times per week\\ 3: Weekly\\ 4:  1-2 times per month\\ 5: Seldom to never}
		\label{fig:question7}
	\end{figure}
		
\end{appendices}
  