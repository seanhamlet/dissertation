\chapter{Conclusions and Future Work}

\section{Summary}
	The overall goal of this project was to optimize respiratory navigator gating to improve the clinical utility of spiral cine DENSE cardiac MRI for the quantification of cardiac mechanics. to accomplish this goal, we completed 3 aims: 1) determined how using a respiratory navigator affects the reproducibility of measures of cardiac mechanics, 2) determined the optimal respiratory navigator gating configuration, and 3) developed and tested an interactive respiratory-controlled videogame for the purpose of improving navigator efficiency during cardiac MRI.

\subsection{Aim 1}
	The purpose of Aim 1 was to understand how using a respiratory navigator during DENSE cardiac MRI can affect the derived cardiac mechanics. Aim 2 was separated into two different studies. In the \textbf{first study}, we examined how strain is affected by inconsistent end-expiratory breath-holds and how using a respiratory navigator could reduce differences and variability in strain. Specifically, we wanted to determine if normal inconsistency in end-expiratory diaphragm position between separate image acquisitions significantly affects estimates of cardiac strains. Strain varies longitudinally throughout the heart \cite{Kuijer2002,Moore2000,Young1994a,Feng2009,NasiraeiMoghaddam2010,Donekal2013a,Suever2017} and patients struggle to hold their breath consistently \cite{Liu1993,Wang1995a,Taylor1997a,Holland1998c,Fischer2006a}. Thus, we hypothesized that inconsistent end-expiratory positions during image acquisition affects the quantification of cardiac strains and therefore results in higher variability in measured strain compared to strains measured at a consistent end-expiratory position by using a respiratory navigator.

	Analysis was performed in 10 healthy volunteers (Age: 22 $\pm$ 6 years, 60\% female) including seven patients with heart disease (Age: 57 $\pm$ 8 years, 43\% female). To simulate end-expiratory position inconsistency, DENSE images were each acquired at the patient-specific minimum, middle, and maximum end-expiratory positions; a repeated acquisition at the middle position was used to quantify variability independent of end-expiratory differences. The range of end-expiratory positions across 10 breath-holds was 10 $\pm$ 4 mm. There were no significant differences in global or regional peak radial, circumferential, or longitudinal strains measured at the different end-expiratory positions (p = 0.17-–0.98). In general, there were also no differences in variability in global or regional peak strains between inconsistent (minimum, middle, and maximum) and consistent (two acquisitions from middle position) end-expiratory positions (p = 0.10-–0.95). In summary, Aim 1 Study 1 demonstrated that \textit{measurements of left ventricular peak strains with DENSE cardiac MR are relatively insensitive to normal changes in end-expiratory position between separate image acquisitions.} Importantly, this indicates that using a respiratory navigator to ensure a consistent end-expiratory position is not required for acquisitions used to derive strains.

	In the \textbf{second study}, we examined how variability in torsion is affected by using a respiratory navigator. Torsion is computed using a basal and apical image acquired during \textit{separate} end-expiratory breath-holds and the assumption that the distance between images remains constant. However, because patients typically struggle to achieve a consistent end-expiratory position for multiple image acquisitions \cite{Liu1993,Wang1995a,Taylor1997a,Holland1998c,Fischer2006a}, this inconsistency in end-expiratory position could lead to variability in the derived torsion measurement. Since torsion has been shown to be limited by high variability \cite{Donekal2013a}, we hypothesized that this variability was partly due to inconsistent end-expiratory positions during serial image acquisition, which could be significantly improved by using a respiratory navigator for cardiac MRI-based quantification of torsion.

	We assessed respiratory-related variability in 2 experiments. In experiment 1, 10 healthy volunteers (Age: 22 $\pm$ 6 years, 60\% female) including seven patients with heart disease (Age: 57 $\pm$ 8 years, 43\% female) underwent DENSE cardiac MRI to compare inter-test variability between consistent and inconsistent end-expiratory positions due to \textit{enforced} end-expiratory position variability. In experiment 2, 20 new, healthy volunteers (Age: 25 $\pm$ 4 years, 60\% female) underwent DENSE cardiac MRI to compare inter-test variability between breath-held and navigator-gated acquisitions to assess variability due to \textit{natural} end-expiratory breath-hold position variability. From experiment 1, enforced variability in end-expiratory position translated to considerable variability in measured torsion (0.56 $\pm$ 0.34$^{\circ}$/cm), whereas inter-test variability with consistent end-expiratory position was 57\% lower (0.24 $\pm$ 0.16$^{\circ}$/cm, p $<$ 0.001). From the second experiment, natural respiratory variability from consecutive breath-holds translated to a variability in torsion of 0.24 $\pm$ 0.10$^{\circ}$/cm, which was significantly higher than the variability from navigator-gated scans (0.18 $\pm$ 0.06$^{\circ}$/cm, p = 0.02). By using a respiratory navigator with DENSE, theoretical sample sizes to detect a clinically meaningful change in torsion were reduced from 66 to 16 and 26 to 15, by using a respiratory navigator, as calculated from the two experiments. Aim 1 Study 2 demonstrated that \textit{a substantial portion (22-57\%) of the inter-test variability of torsion can be reduced by using a respiratory navigator to ensure a consistent breath-hold position between image acquisitions.}

\subsection{Aim 2}
	The purpose of Aim 2 was to determine the optimal respiratory navigator gating configuration for the quantification of left ventricular strain using spiral cine DENSE MRI. Two-dimensional spiral cine DENSE was performed using two single-navigator configurations (retrospective, prospective) and a combined “dual-navigator” configuration in 10 healthy adults (Age: 23 $\pm$ 3 years, 40\% female) and 20 healthy children  (Age: 13 $\pm$ 3 years, 45\% female. The adults also underwent breath-hold DENSE as a reference standard for comparisons. Peak left ventricular strains, signal-to-noise ratio (SNR), and navigator efficiency were compared. Subjects also underwent dual-navigator gating with and without visual feedback to determine the effect on navigator efficiency. There were no differences in circumferential, radial, and longitudinal strains between navigator-gated and breath-hold DENSE (P = 0.09-–0.95). The dual configuration maintained SNR compared with breath-hold acquisitions (16 versus 18, P = 0.06). SNR for the prospective configuration was lower than for the dual navigator in adults (P = 0.004) and children (P $<$ 0.001). Navigator efficiency was higher (P $<$ 0.001) for both retrospective (54\%) and prospective (56\%) configurations compared with the dual configuration (35\%). Visual feedback improved the dual configuration navigator efficiency to 55\% (P $<$ 0.001). Aim 2 demonstrated: \textit{1) when quantifying left ventricular strains using spiral cine DENSE MRI, a dual navigator configuration results in the highest SNR in adults and children, 2) in adults, a retrospective configuration has good navigator efficiency without a substantial drop in SNR, 3) prospective gating should be avoided because it has the lowest SNR, and 4) visual feedback represents an effective option to maintain navigator efficiency while using a dual navigator configuration.}

\subsection{Aim 3}
	The purpose of Aim 3 was to develop and test an interactive videogame to improve navigator efficiency in children undergoing DENSE cardiac MRI. Advanced cardiac MRI acquisitions often require long scan durations that necessitate respiratory navigator gating. The tradeoff of navigator gating is reduced scan efficiency, particularly when the patient's breathing patterns are inconsistent, as is commonly seen in children. We hypothesized that engaging pediatric participants with a navigator-controlled videogame to help control breathing patterns would improve navigator efficiency and maintain image quality. We developed custom software that processed the Siemens respiratory navigator image in real-time during CMR and represented diaphragm position using a cartoon avatar, which was projected to the participant in the scanner as visual feedback. The game incentivized children to breathe such that the avatar was positioned within the navigator acceptance window ($\pm$3 mm) throughout image acquisition. Fifty children (Age: 14 $\pm$ 3 years, 48\% female) with no significant past medical history underwent a respiratory navigator-gated 2D spiral cine DENSE cardiac MRI acquisition first with no feedback (NF) and then with the feedback game (FG). Thirty of the 50 children were randomized to undergo extensive off-scanner training with the FG using a MRI simulator, or no off-scanner training. Navigator efficiency, signal-to-noise ratio (SNR), and global left-ventricular strains were determined for each participant and compared. Using the FG improved average navigator efficiency from 33 $\pm$ 15 to 58 $\pm$ 13\% (P $<$ 0.001) and improved SNR by 5\% (P = 0.01) compared to acquisitions with NF. There was no difference in navigator efficiency (P = 0.90) or SNR (P = 0.77) between untrained and trained participants for FG acquisitions. Circumferential and radial strains derived from FG acquisitions were slightly reduced compared to NF acquisitions (−16 $\pm$ 2\% vs −17 $\pm$ 2\%, P $<$ 0.001; 40 $\pm$ 10\% vs 44 $\pm$ 11\%, P = 0.005, respectively). There were no differences in longitudinal strain (P = 0.38). Aim 3 demonstrated that \textit{use of a respiratory navigator feedback game during navigator-gated CMR improved navigator efficiency in children from 33 to 5\%. This improved efficiency was associated with a 5\% increase in SNR for spiral cine DENSE. Extensive off-scanner training was not required to achieve the improvement in navigator efficiency.}

\section{Clinical Implications}
	The goal of this project was to optimize respiratory navigator gating for use during DENSE cardiac MRI. From Aim 1 Study 1, it was demonstrated that the quantification of peak left ventricular cardiac strains is relatively insensitive to normal variations in end-expiratory positions between image acquisitions. In the clinical setting, since there were no differences in peak strain between end-expiratory positions, patient end-expiratory diaphragm position does not have to be monitored when performing breath-hold DENSE acquisition for single image analyses. These findings should generalize to other image acquisitions that are used to derive measures of cardiac strains.

	However, Aim 1 Study 2 demonstrated that using a respiratory navigator significantly improves the variability of measured torsion. Thus, where possible, a respiratory navigator should be employed for acquisition of LV torsion data to minimize variability. For LV torsion, if inconsistency in end-expiratory position is not taken care of during scans, then it is important to understand its effects, which will dramatically increased study sample sizes for research and reduce the ability to detect meaningful differences in individual patients.

	From Aim 2, the dual-navigator, among the navigator configurations, has been shown to result in the best image quality in both adults and children. It's important to understand the limitations of the dual-navigator before employing in clinical practice, however, due to its worse navigator efficiency compared to other navigator configurations. Therefore, some form of visual feedback (of the diaphragm position) should be employed, where possible, in order to achieve an adequate scan duration.

	From Aim 3, it was demonstrated that using an interactive feedback videogame during DENSE cardiac MRI substantially improved navigator efficiency. Importantly, 1) minimal equipment is needed in order to use the videogame and 2) the equipment does not directly integrate into an imaging sequence; it connects externally to the scanner user interface. Thus, the videogame can easily be adopted at research and clinical sites that utilize navigator-gated cardiac MRI acquisitions, especially in children. Directly related to the overall goal of this project, since using the videogame results in improved navigator efficiency, as well as reduced acquisition times, use of the videogame can help improve the clinical feasibility of advanced imaging techniques, such as DENSE. Besides reducing acquisition time, which saves time, increased navigator efficiency can also be used to acquire more data, such as to improve spatial or temporal resolution of images [cite]. 

	Notably, off-scanner training was not required in order to achieve the improved navigator efficiency by using the videogame. Therefore, clinical and research sites would not have to invest in resources for building a simulator and in time for training children prior to undergoing navigator-gated cardiac MRI acquisitions. Also, this study was performed only in children, but since previous studies have shown that navigator feedback can improve navigator efficiency in adults, the videogame should also work well in adults [cite]. The findings of this study are definitely applicable to cardiac DENSE MRI, but it is likely that these findings generalize to several cardiac MRI techniques that use a respiratory navigator. Higher resolution imaging, which may be more sensitive to registration issues, may not result in the same findings. An example would be coronary MR angiography.

\section{Future Directions}

\subsection{Aim 1}
	Chapters 2 and 3 demonstrated that variability torsion, but not strain, could be significantly imporved by using a respiratory navigator to ensure a consistent position between image acquisition.

	% chapter 2
	We used respiratory navigator gating to acquire the DENSE cardiac images, which reduces respiratory artifacts during image acquisition, so we could not measure the effect of inconsistent end-expiratory position during breath-holds on the derived strains. It would be beneficial to quantify the amount of end-expiratory position variability during breath-hold cardiac MR image acquisition and determine whether the magnitude of inconsistent end-expiratory positions correlates with changes in strain values. An example would be to explore whether inconsistent end-expiratory positions during a breath-hold DENSE scan causes blurring due to motion and results in lower strain magnitudes.
	This study examined the effects of inconsistent end-expiratory positions on cardiac strains in a small patient sample. It would be beneficial to investigate this effect in a larger patient sample who have heterogeneous contraction patterns, for example, due to post-myocardial infarction. These patients may have steeper gradients in strain across infarcted to non-infarcted tissue regions [39]. Therefore, we cannot definitively say that the effects of inconsistent end-expiratory positions in that setting are similarly small and negligible. Future studies should investigate strain variability due to inconsistent end-expiratory positions in patients who have infarcted tissue in specific regions (e.g. anterior vs inferior).

	% chapter 3
	We examined the effects of variable end-expiratory position on LV torsion in a small patient sample. It may be beneficial to examine these results in a larger, more heterogeneous patient sample to determine whether specific diseases affect the results more than others, especially conditions that affect a patient’s ability to repeatedly hold his or her breath reproducibly (for example, pulmonary diseases). Due to the lengthy duration of DENSE breath-holds
	(~20 s) and their limitations in breath-holding ability, the breath-hold acquisition protocol was not performed in patients. Based on these factors, we expect that pa- tients would demonstrate higher variability in LV torsion with the breath-hold measures compared to the healthy volunteers we studied. Hence, the potential reduction in mean variability when using the respiratory navigator may in fact be higher than the 22% we report from the healthy volunteers in experiment 2. Nevertheless, the re- duction in LV torsion variability patients will achieve by using a respiratory navigator will likely fall between the study’s reported values of 22 and 57%.

	% combo chapters 2 and 3 future directions

\subsection{Aim 2}
	% rewrite
	A third limitation of this study is the lack of assess-
	ment of clinical patients. Cardiac patients, who routinely undergo MR imaging and who may have limited ability to hold their breath, may not be able to perform the lengthy breathhold scan and may not achieve as high navigator effi- ciency when performing a dual navigator scan with feed- back. However, because this population is more likely to undergo DENSE MR imaging than this study’s volunteer subjects, it would be beneficial to determine whether the results remain the same. For example, it may be important to use dual navigator gating, even at the expense of naviga- tor efficiency, to achieve higher SNR, because SNR is com- monly lower in the clinical patient population compared with healthy volunteers.

\subsection{Aim 3}
	% re-write this
	We performed this study in children with no significant past medical history. While we did attempt to re- cruit from a broad clinical population using recruitment services at our Center for Clinical and Translational Science, the population we ultimately studied may not be entirely representative of a standard pediatric clinical population that would routinely undergo cardiac MRI. For example, approximately 25\% of patients with tetral- ogy of Fallot may have learning and behavioral difficul- ties [23], which may impair their ability to benefit from the feedback game. It is therefore reasonable to expect that the true benefit of the feedback game in a standard clinical population will be smaller than what was mea- sured in the current study, but still better than what can be expected without the use of feedback. Even if only half of the patients benefit to the extent shown in the current study, the overall navigator efficiency for the clinical population as a whole would still increase from 33 % efficiency to 46 % efficiency (a 38 % relative bene- fit). Future research will seek to evaluate this in further detail as we implement the feedback game during rou- tine clinical workflows.
	
	Further study is needed to test higher resolution techniques, instead of DENSE, for these applications.
	
	Distribute videogame to research sites. have them use in patients since we only measure things in healthy subects
	
	- children who have duchense muscular distrophy
	- adults with heart disease and failure
	see if same conclusions hold.
	
	re-design videogame using more general, free (and not expensive license-based software) so that it can be distrbuted easier
	
	incorporate videogame into siemens scanner itself as an option instead of regular respiratory navigator
	
\subsection{Final Thoughts}
	
	