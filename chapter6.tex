\chapter{Conclusions and Future Work}

\section{Summary}
	The overall goal of this project was to optimize respiratory navigator gating to improve the clinical utility of spiral cine DENSE cardiac MRI for the quantification of cardiac mechanics. To accomplish this goal, we completed 3 aims: 1) determined how using a respiratory navigator affects the reproducibility of measures of cardiac mechanics, 2) determined the optimal respiratory navigator gating configuration, and 3) developed and tested an interactive respiratory-controlled videogame to improve navigator efficiency during cardiac MRI.

\subsection{Aim 1}
	The purpose of Aim 1 was to understand how using a respiratory navigator during DENSE cardiac MRI can affect the derived cardiac mechanics. Aim 1 was separated into two different studies. In the \textbf{first study}, we examined how the measurement of cardiac strain is affected by inconsistent end-expiratory breath-holds and how using a respiratory navigator could reduce differences and variability in strain. Specifically, we wanted to determine if normal inconsistency in end-expiratory diaphragm position between separate image acquisitions significantly affects estimates of cardiac strains. Strain varies longitudinally throughout the heart \cite{Kuijer2002,Moore2000,Young1994a,Feng2009,NasiraeiMoghaddam2010,Donekal2013a,Suever2017} and patients struggle to hold their breath consistently \cite{Liu1993,Wang1995a,Taylor1997a,Holland1998c,Fischer2006a}. Thus, we hypothesized that inconsistent end-expiratory positions during image acquisition affects the quantification of cardiac strains and therefore results in higher variability in measured strain compared to strains measured at a consistent end-expiratory position by using a respiratory navigator.

	Analysis was performed in 10 healthy volunteers (Age: 22 $\pm$ 6 years, 60\% female) including seven patients with heart disease (Age: 57 $\pm$ 8 years, 43\% female). To simulate end-expiratory position inconsistency, DENSE images were each acquired at the patient-specific minimum, middle, and maximum end-expiratory positions; a repeated acquisition at the middle position was used to quantify variability independent of end-expiratory differences. The range of end-expiratory positions across 10 breath-holds was 10 $\pm$ 4 mm. There were no significant differences in global or regional peak radial, circumferential, or longitudinal strains measured at the different end-expiratory positions (p = 0.17-–0.98). In general, there were also no differences in variability in global or regional peak strains between inconsistent (minimum, middle, and maximum) and consistent (two acquisitions from middle position) end-expiratory positions (p = 0.10-–0.95). In summary, Aim 1 Study 1 demonstrated that \textit{measurements of left ventricular peak strains with DENSE cardiac MR are relatively insensitive to normal changes in end-expiratory position between separate image acquisitions.} Importantly, this indicates that using a respiratory navigator to ensure a consistent end-expiratory position is \textit{not} required for acquisitions used to derive cardiac strains.

	In the \textbf{second study}, we examined how variability in the quantification of left ventricular torsion is affected by using a respiratory navigator. Torsion is computed using a basal and apical image acquired during \textit{separate} end-expiratory breath-holds and the assumption that the distance between the acquired images is precisely known. However, because patients typically struggle to achieve a consistent end-expiratory position for multiple image acquisitions \cite{Liu1993,Wang1995a,Taylor1997a,Holland1998c,Fischer2006a}, this inconsistency in end-expiratory position could lead to variability in the measurement of torsion. Since torsion has been shown to be limited by high variability \cite{Donekal2013a}, we hypothesized that this variability was partly due to inconsistent end-expiratory positions during serial image acquisition, which could be significantly improved by using a respiratory navigator.

	We assessed respiratory-related variability in 2 experiments. In experiment 1, 10 healthy volunteers (Age: 22 $\pm$ 6 years, 60\% female) including seven patients with heart disease (Age: 57 $\pm$ 8 years, 43\% female) underwent DENSE cardiac MRI to compare inter-test variability between consistent and inconsistent end-expiratory positions due to \textit{enforced} end-expiratory position variability. In experiment 2, twenty new, healthy volunteers (Age: 25 $\pm$ 4 years, 60\% female) underwent DENSE cardiac MRI to compare inter-test variability between breath-held and navigator-gated acquisitions to assess variability due to \textit{natural} end-expiratory breath-hold position variability. From experiment 1, enforced variability in end-expiratory position translated to considerable variability in measured torsion (0.56 $\pm$ 0.34 $^{\circ}$/cm), whereas inter-test variability with consistent end-expiratory position was 57\% lower (0.24 $\pm$ 0.16 $^{\circ}$/cm, p $<$ 0.001). From the second experiment, natural respiratory variability from consecutive breath-holds translated to a variability in torsion of 0.24 $\pm$ 0.10 $^{\circ}$/cm, which was significantly higher than the variability from navigator-gated scans (0.18 $\pm$ 0.06 $^{\circ}$/cm, p = 0.02). By using a respiratory navigator with DENSE, theoretical sample sizes to detect a clinically meaningful change in torsion were reduced from 66 to 16 and 26 to 15, by using a respiratory navigator, as calculated from the two experiments. Aim 1 Study 2 demonstrated that \textit{a substantial portion (22-57\%) of the inter-test variability of torsion can be reduced by using a respiratory navigator to ensure a consistent breath-hold position between image acquisitions.}

\subsection{Aim 2}
	The purpose of Aim 2 was to determine the optimal respiratory navigator gating configuration for the quantification of left ventricular strain using spiral cine DENSE MRI. Two-dimensional spiral cine DENSE was performed using two single-navigator configurations (retrospective, prospective) and a combined “dual-navigator” configuration in 10 healthy adults (Age: 23 $\pm$ 3 years, 40\% female) and 20 healthy children  (Age: 13 $\pm$ 3 years, 45\% female). The adults also underwent breath-hold DENSE as a reference standard for comparisons. Peak left ventricular strains, signal-to-noise ratio (SNR), and navigator efficiency were compared. Subjects also underwent dual-navigator gating with and without visual feedback to determine the effect on navigator efficiency. There were no differences in circumferential, radial, and longitudinal strains between navigator-gated and breath-hold DENSE (p = 0.09-–0.95). The dual configuration maintained SNR compared with breath-hold acquisitions (16 versus 18, p = 0.06). SNR for the prospective configuration was lower than for the dual navigator in adults (p = 0.004) and children (p $<$ 0.001). Navigator efficiency was higher (p $<$ 0.001) for both retrospective (54\%) and prospective (56\%) configurations compared with the dual configuration (35\%). Visual feedback improved the dual configuration navigator efficiency to 55\% (p $<$ 0.001). Aim 2 demonstrated when quantifying left ventricular strains using spiral cine DENSE MRI, \textit{1) a dual navigator configuration results in the highest SNR in adults and children, 2) in adults, a retrospective configuration has good navigator efficiency without a substantial drop in SNR, 3) prospective gating should be avoided because it has the lowest SNR, and 4) visual feedback represents an effective option to maintain navigator efficiency while using a dual navigator configuration.}

\subsection{Aim 3}
	The purpose of Aim 3 was to develop and test an interactive videogame designed to improve navigator efficiency in children undergoing DENSE cardiac MRI. Advanced cardiac MRI acquisitions often require long scan durations that necessitate respiratory navigator gating. The tradeoff of navigator gating is reduced scan efficiency, particularly when the patient's breathing patterns are inconsistent, as is commonly seen in children. We hypothesized that engaging pediatric participants with a navigator-controlled videogame to help control breathing patterns would improve navigator efficiency and maintain image quality. We developed custom software that processed the Siemens respiratory navigator image in real-time during CMR and represented diaphragm position using a cartoon avatar, which was projected to the participant in the scanner as visual feedback. The game incentivized children to breathe such that the avatar was positioned within the navigator acceptance window ($\pm$3 mm) throughout image acquisition. Fifty children (Age: 14 $\pm$ 3 years, 48\% female) without significant past medical history underwent a respiratory navigator-gated 2D spiral cine DENSE cardiac MRI acquisition first without feedback and then with the feedback videogame. Thirty of the 50 children were randomized to undergo off-scanner training with the videogame using a MRI simulator, or no off-scanner training. Navigator efficiency, SNR, and global left-ventricular strains were determined for each participant and compared. Using the videogame improved average navigator efficiency from 33 $\pm$ 15 to 58 $\pm$ 13\% (p $<$ 0.001) and improved SNR by 5\% (p = 0.01) compared to acquisitions without feedback. There was no difference in navigator efficiency (p = 0.90) or SNR (p = 0.77) between untrained and trained participants for videogame acquisitions. Circumferential and radial strains derived from videogame acquisitions were slightly reduced compared to no feedback acquisitions (−16 $\pm$ 2\% vs −17 $\pm$ 2\%, p~$<$ 0.001; 40 $\pm$ 10\% vs 44 $\pm$ 11\%, p = 0.005, respectively). There were no differences in longitudinal strain (p = 0.38). Aim 3 demonstrated that \textit{use of a respiratory navigator feedback videogame during navigator-gated CMR improved navigator efficiency in children from 33 to 58\%. This improved efficiency was associated with a 5\% increase in SNR for spiral cine DENSE. Off-scanner training was not required to achieve the improvement in navigator efficiency.}

\section{Clinical Implications}
	The goal of this project was to optimize respiratory navigator gating for use during DENSE cardiac MRI. From Aim 1 Study 1, it was demonstrated that the quantification of peak left ventricular cardiac strains was relatively insensitive to normal variations in end-expiratory positions between image acquisitions. In the clinical setting, since there were no differences in peak strain between end-expiratory positions, patient end-expiratory diaphragm position does not have to be monitored when performing breath-hold DENSE acquisition for single image analyses. These findings should generalize to other image acquisitions that are used to derive measures of cardiac strains.

	However, Aim 1 Study 2 demonstrated that using a respiratory navigator significantly improves the variability of measured left ventricular torsion. Thus, where possible, a respiratory navigator should be employed for acquisition of left ventricular torsion data to minimize variability. For torsion, if inconsistency in end-expiratory position is not taken care of during scans, then it is important to understand its effects, which will lead to dramatically increased study sample sizes for research and reduce the ability to detect meaningful differences in torsion in individual patients.

	From Aim 2, the dual-navigator configuration has been shown to result in the best image quality in both adults and children. It is important to understand the limitations of the dual-navigator before employing it in clinical practice, however, due to its worse navigator efficiency compared to other navigator configurations. Therefore, some form of visual feedback (of the diaphragm position) should be used, where possible, to achieve an adequate scan duration along with the improved image quality.

	From Aim 3, it was demonstrated that using an interactive feedback videogame during DENSE cardiac MRI substantially improved navigator efficiency. Importantly, 1) minimal equipment is needed to implement the videogame and 2) the equipment does not directly integrate into an imaging sequence; it connects externally to the scanner user interface. Thus, the videogame can easily be adopted at research and clinical sites that utilize navigator-gated cardiac MRI acquisitions, especially in children. Since using the videogame results in improved navigator efficiency, which lead to reduced acquisition times, use of the videogame can help improve the clinical feasibility of advanced imaging techniques, such as DENSE. Besides reducing acquisition time, which saves time, increased navigator efficiency can also be used to acquire more data, such as to improve spatial or temporal resolution of images \cite{Feuerlein2009}. 

	Notably, off-scanner training was not required to achieve the improved navigator efficiency by using the videogame. Therefore, clinical and research sites would not have to invest in resources for building a simulator or spend significant time training children prior to undergoing navigator-gated cardiac MRI acquisitions. This study was performed only in children, but since previous studies have shown that navigator feedback can improve navigator efficiency in adults \cite{Feuerlein2009,Liu1993}, the videogame should also work well in adults. The findings of this study are definitely applicable to cardiac DENSE MRI, but it is likely that these findings generalize to several cardiac MRI techniques that use a respiratory navigator. Higher resolution imaging, which may be more sensitive to registration issues, may not result in the same findings. An example would be coronary MR angiography.

\section{Future Directions}

\subsection{Aim 1}
	In Chapter 2, we used a respiratory navigator to measure and acquire DENSE cardiac images. Since a respiratory navigator restricts data acquisition to a narrow window, we could not measure the effect of inconsistent end-expiratory position \textit{during} breath-holds on the derived cardiac strains. It would be important to quantify variability in end-expiratory position during breath-hold DENSE cardiac MRI acquisition and to examine whether the changes in strain values correspond to the magnitude of variability in end-expiratory position.
	
	In Chapters 2 and 3, we examined differences and variability in strains and torsion due to inconsistent breath-hold positions in a relatively small patient population. It would be beneficial to see if these results remain the same for a larger patient group with more heterogeneous contraction patterns, for example in patients with a history of myocardial infarction that has affected contractile function within a specific region of the heart. In particular, patients who have infarcted regions may have steeper gradients in strain from an infarcted to non-infarcted region. Moreover, it would be interesting to investigate whether certain diseases affect the results more than others, such as conditions that affect a patient's ability to breath-hold in a consistent manner.

\subsection{Aim 2}
	For Chapter 4, we identified the optimal navigator configuration using \textit{healthy} subjects. Due to the lengthy breath-hold scans and patients' limited ability to breath-hold, they may not be able to perform the long breath-hold acquisition. Moreover, patients may not be able to achieve as high of a navigator efficiency as the healthy subjects did in this study. Therefore, it would be important to determine if the results of the optimal navigator configuration remain the same when using patients.

\subsection{Aim 3}
	We only tested the videogame in children without significant past medical history. Similar to future directions of Chapters 2 through 4, Chapter 5's study should also be performed in children who routinely undergo cardiac MRI acquisitions. In particular, it is known that 25\% of patients born with Tetrology of Fallot (a congenital heart disease) have behavioral and learning difficulties \cite{Piran2011}, which can make it challenging for them to benefit from using the videogame. However, even if only a portion of the patients achieve better than typical navigator efficiencies, there would still be clinical benefit.

	Another future direction would be to set up the videogame at other research sites, particularly sites that have a high throughput of patients who undergo cardiac MRI acquisitions. Due to its non-invasive connection to the scanner, the videogame can be easily adopted. Recently, the videogame has been implemented in a different MRI acquisition platform (Philips) at Boston Children's Hospital to test its efficacy in a patient population. Instead of a navigator image to capture, this Philips acquisition platform had a continuously updating text file containing the required locations (acceptance window and diaphragm position). Due to the modular nature of the code, all that was needed was to write a new input source class that accepted an updating log file instead of a navigator image.

	In particular, if new features are needed in order to adapt the videogame to a different scanner or for patients to better understand or benefit from the videogame, they can be easily added.
	
	Right now, the videogame is written using MATLAB, which requires the purchase of a license. Another possible future direction would be to re-design the videogame using free programming software so that it can be more freely distributed and won't be limiting to people who do not have the MATLAB software license. 
	
	Lastly, it would be beneficial to eventually incorporate the videogame as a viewing option within the scanner software itself. This would make the videogame even more adoptable at clinical and research sites. Collaborations with MRI vendors such as Siemens or Philips would be required to achieve this goal, and are currently being investigated [TODO: come up with bigger ideas for future directions!].
	
	